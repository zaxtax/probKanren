%% The first command in your LaTeX source must be the \documentclass command.
%%
%% Options:
%% twocolumn : Two column layout.
%% hf: enable header and footer.
\documentclass[
% twocolumn,
% hf,
]{ceurart}

%%
%% One can fix some overfulls
% \sloppy

%%
%% Minted listings support 
%% Need pygment <http://pygments.org/> <http://pypi.python.org/pypi/Pygments>
\usepackage{minted}
%% auto break lines
\setminted{breaklines=true}

%%
%% end of the preamble, start of the body of the document source.
\begin{document}

%%
%% Rights management information.
%% CC-BY is default license.
\copyrightyear{2021}
\copyrightclause{Copyright for this paper by its authors.
  Use permitted under Creative Commons License Attribution 4.0
  International (CC BY 4.0).}

%%
%% This command is for the conference information
\conference{Woodstock'21: Symposium on the irreproducible science,
  June 07--11, 2021, Woodstock, NY}

%%
%% The "title" command
\title{probKanren: A Simple Probabilistic extension for miniKanren}

%%
%% The "author" command and its associated commands are used to define
%% the authors and their affiliations.
\author[1]{Robert Zinkov}[%
%orcid=0000-0002-0877-7063,
email=zinkov@robots.ox.ac.uk,
url=https://zinkov.com/,
]
\address[1]{Peoples' Friendship University of Russia (RUDN University),
  6 Miklukho-Maklaya St, Moscow, 117198, Russian Federation}

\author[2]{William E. Byrd}[%
%orcid=0000-0001-7116-9338,
email=webyrd@uab.edu,
url=http://webyrd.net/,
]
\address[2]{Hugh Kaul Precision Medicine Institute, University of Alabama at Birmingham, 705 20th Street S., Birmingham, AL 35233, United States of America}

%%
%% The abstract is a short summary of the work to be presented in the
%% article.
\begin{abstract}
  Probabilistic programming is sometimes described as a generalisation
  of logic programming where instead of just returning a set of facts
  that answer a given query, we return a probability distribution over
  facts that answer a given query. But many contemporary probabilistic
  logic programming languages are not implemented as simple extensions
  of existing logic programming languages but instead involve their
  own unique implementations. Here we introduce probKanren, a simple
  extension to miniKanren that transforms
\end{abstract}

%%
%% Keywords. The author(s) should pick words that accurately describe
%% the work being presented. Separate the keywords with commas.
\begin{keywords}
  Probabilistic Logic Programming \sep
  miniKanren \sep
  Probabilistic Programming \sep
  Sequential Monte Carlo
  %Particle Filtering
  %relational programming
  %logic programming
\end{keywords}

%%
%% This command processes the author and affiliation and title
%% information and builds the first part of the formatted document.
\maketitle

\section{Introduction}

Conceptually, logic programming provides a way to model
non-determinism. This is accomplished by maintaining a set of answers
that satisfy a set of logical constraints. A natural generalisation
to this domain is adding a notion of uncertainty to this set of answers
by associating with them a probability distribution.

\subsection{Illustrated Example}

To help explain how to use probKanren we introduce the following example:

\begin{minted}{Scheme}
  (run 1000 (q)
    (normal 0 3 q)
    (greatero q 0))
\end{minted}

This probKanren program draws 1000 samples from a normal distribution truncated
below 0.

\subsection{Contributions}

\section{Related Work}

There is a rich history of extending logic programming formalisms to
support probabilistic inference. Early systems like
PRISM\cite{sato1997prism} and ProbLog\cite{de2007problog} allowed
associating discrete distributions with facts. Later
work\cite{gutmann2010extending} introduces distribution clauses so
that some continuous distributions. Other work extended these methods
further while focusing on efficient exact inference algorithms like
model weight integration\cite{islam2012inference,
  belle2015probabilistic}. Our work is most similar to
\cite{gutmann2011magic} except that while they combine their forward
reasoning with an importance sampler we use a particle cascade instead
which can be more sample efficient.

\section{Background}

\subsection{miniKanren}

\textsc{miniKanren}\cite{byrd2017unified, daniel2018reasoned} is a
pure logic programming language embedded in Scheme. The language
consists of a set of terms, functions for constructing, a set of goal
primitives, and two run functions to answer queries in the
language. The goal primitives consist of a \texttt{fresh} form for
introducing logic variables, a unification primitive, a conjunction
combinator, and a disjunction combinator. 

\subsection{Probabilistic Programming}

Probabilistic Programming Languages\cite{wood2014new,
  van2018introduction} are a family of domain specific languages for
posing and efficiently solving probabilistic modelling problems. At
their core, all have a way to \texttt{sample} from a probability
distribution and \texttt{observe} data generated from a probability
distribution.

There are many ways to implement inference algorithms for probabilistic
programming languages but methods based on likelihood-weighting and
sequential monte carlo algorithms are the easiest.

\subsection{Sequential Monte Carlo}

Sequential Monte Carlo\cite{chopin2020introduction} are an efficient
online way to sample from probabilistic models especially suited for
state-space domains. If we imagine our probabilistic programs as
straight-line programs with no control-flow we can imagine numbering
every sample function $x_1, f_2, \ldots, f_n$ and every observe
function $g_1, g_2, \ldots, g_n$ then our probability density over our
random variables $x$ and observed data $y$ can be defined as:

\begin{equation}
  p(x, y) = \prod_i f(x_i \mid x_{0:n-1})g(y_i \mid x_{0:n})
\end{equation}

\subsection{Sequential Importance Sampling}

If we imagine running this program each execution trace can be seen as
a sample from the distribution. If we then weight by their likelihood
as we run them that collection execution traces will be an empirical
distribution of the given program. We call each of these execution traces
particles and the below method is how we obtain them along with their weights $w$.

\begin{equation}
  x_n \sim f(x_n \mid x_{0:n-1}) \\
  w_n^{(k)} = \frac{g(x_n \mid x_{0:n})f(x_n \mid x_{0:n-1})}{q(x_n \mid x_{0:n-1})} \\
  W_n^{(k)} = W_{n-1}^{(k)} w_n^{(k)}
\end{equation}

The weights are then normalised at the end to make a proper probability distribution.

\subsection{Sequential Importance Resampling}

The problem with the above algorithm is over time for many particles
$W_k$ is going to become low and that particle will stop being very
informative of the underlying distribution. To mitigate this issue,
each time we encounter an observation we resample our particles. Resampling
effectively removes particles with low weight and duplicates particles
with higher weight by sampling with replacement our existing particles.

The above is called multinomial resampling but there are other methods
as well. A survey\cite{douc2005comparison} of resampling methods
suggests all of them are helpful to reduce particle degeneracy.

\subsection{Particle Cascade}

As the SMC resampling step was defined in the previous section
Particle Cascades \cite{PaigeWDT14} remove this barrier allowing
every particle to be resampled asynchronously with the associated
weights being relative to a global running average. % double-check this

\section{Proposed Method}

We propose to extend \textsc{miniKanren} by augmenting each of the
search streams with a set of particles. These particles represent the
empirical distribution of that stream. Each particle has associated
with it a substitution of all the logic and random variables as well
as a weight that is proportional to the likelihood of the substitution.

An initial set of particles is created from the probabilistic program
when it is first run. As disjunctions like \texttt{conde} are encountered,
we split evenly the number of particles allocated to each stream. Whenever
we encounter a unification primitive, we run a resampling step. This
helps to prune low-weight particles and replicate high weight ones.

As an optimisation we may create more particles during resampling
based on a globally stored a counter of the effective sample size of
all particles across all streams.

We follow \cite{gutmann2010extending} and place the following restrictions
on our distribution clauses and the random variables they specify.

Firstly, the arguments of distribution clauses must be
ground. Secondly, a random variable can not unify with any arithmetic
expression


\section{Experiments}

One of the experiments should be something from the probabilistic logic literature.

Friends who Smoke is a probabilistic logic program which models the social nature of
who smokes cigarettes. The model predicts that people who are friends with people
who smoke are more likely to smoke.

We could include experiment from Noah Goodman's dippl work like a semantic parsing
example\\ \url{http://dippl.org/examples/semanticparsing.html}

We could include program synthesis experiment, where probabilities
allow us to specify a soft preference for using certain language
primitives in a similar spirit to
\textsc{rKanren}\cite{swords2013rkanren} or neural-guided search
\cite{zhang2018neural}.

\section{Conclusions}

We made a cool and simple to implement extension to
\textsc{miniKanren} that let's us support probabilistic inference on
both discrete and continuous distributions.

%% \begin{table*}
%%   \caption{Frequency of Special Characters}
%%   \label{tab:freq}
%%   \begin{tabular}{ccl}
%%     \toprule
%%     Non-English or Math&Frequency&Comments\\
%%     \midrule
%%     \O & 1 in 1,000& For Swedish names\\
%%     $\pi$ & 1 in 5& Common in math\\
%%     \$ & 4 in 5 & Used in business\\
%%     $\Psi^2_1$ & 1 in 40,000& Unexplained usage\\
%%   \bottomrule
%% \end{tabular}
%% \end{table*}

%% To set a wider table, which takes up the whole width of the page's
%% live area, use the environment \verb|table*| to enclose the table's
%% contents and the table caption.  As with a single-column table, this
%% wide table will ``float'' to a location deemed more
%% desirable. Immediately following this sentence is the point at which
%% Table~\ref{tab:commands} is included in the input file; again, it is
%% instructive to compare the placement of the table here with the table
%% in the printed output of this document.

%% \begin{table}
%%   \caption{Some Typical Commands}
%%   \label{tab:commands}
%%   \begin{tabular}{ccl}
%%     \toprule
%%     Command &A Number & Comments\\
%%     \midrule
%%     \texttt{{\char'134}author} & 100& Author \\
%%     \texttt{{\char'134}table}& 300 & For tables\\
%%     \texttt{{\char'134}table*}& 400& For wider tables\\
%%     \bottomrule
%%   \end{tabular}
%% \end{table}


%% The ``\verb|figure|'' environment should be used for figures. One or
%% more images can be placed within a figure. If your figure contains
%% third-party material, you must clearly identify it as such, as shown
%% in the example below.
%% \begin{figure}
%%   \centering
%%   %\includegraphics[width=\linewidth]{sample-franklin}
%%   \caption{1907 Franklin Model D roadster. Photograph by Harris \&
%%     Ewing, Inc. [Public domain], via Wikimedia
%%     Commons. (\url{https://goo.gl/VLCRBB}).}
%% \end{figure}

%% Define the bibliography file to be used
\bibliography{main}

%%
%% If your work has an appendix, this is the place to put it.
\appendix

\end{document}
